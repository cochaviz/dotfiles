% Created 2022-12-05 ma 21:25
% Intended LaTeX compiler: pdflatex
\documentclass[11pt]{article}
\usepackage[utf8]{inputenc}
\usepackage[T1]{fontenc}
\usepackage{graphicx}
\usepackage{longtable}
\usepackage{wrapfig}
\usepackage{rotating}
\usepackage[normalem]{ulem}
\usepackage{amsmath}
\usepackage{amssymb}
\usepackage{capt-of}
\usepackage{hyperref}
\author{Zohar Cochavi}
\date{\today}
\title{}
\hypersetup{
 pdfauthor={Zohar Cochavi},
 pdftitle={},
 pdfkeywords={},
 pdfsubject={},
 pdfcreator={Emacs 28.2 (Org mode 9.6)}, 
 pdflang={English}}
\usepackage{biblatex}
\addbibresource{/home/zohar/.dotfiles//ucs.bib}
\begin{document}

\tableofcontents

;;; \$DOOMDIR/config.el -\textbf{- lexical-binding: t; -}-

;; Place your private configuration here! Remember, you do not need to run 'doom
;; sync' after modifying this file!


;; Some functionality uses this to identify you, e.g. GPG configuration, email
;; clients, file templates and snippets.
(setq user-full-name ``Zohar Cochavi''
      user-mail-address ``cochavi.zohar@protonmail.com'')

;; Doom exposes five (optional) variables for controlling fonts in Doom. Here
;; are the three important ones:
;;
;; + `doom-font'
;; + `doom-variable-pitch-font'
;; + `doom-big-font' -- used for `doom-big-font-mode'; use this for
;;   presentations or streaming.
;;
;; They all accept either a font-spec, font string (``Input Mono-12''), or xlfd
;; font string. You generally only need these two:
;; (setq doom-font (font-spec :family ``monospace'' :size 12 :weight 'semi-light)
;;       doom-variable-pitch-font (font-spec :family ``sans'' :size
(setq doom-font (font-spec :family ``Iosevka Extended'' :size 14)
      doom-variable-pitch-font(font-spec :family ``Iosevka Aile'' :size 14 :weight 'light)
      doom-big-font (font-spec :family ``Fira Mono'' :size 24))

(custom-set-faces!
  '(font-lock-comment-face :slant italic)
  '(font-lock-keyword-face :slant italic))

;; There are two ways to load a theme. Both assume the theme is installed and
;; available. You can either set `doom-theme' or manually load a theme with the
;; `load-theme' function. This is the default:

(setq doom-theme 'doom-xresources)
(setq doom-themes-padded-modeline t)


;; If you use `org' and don't want your org files in the default location below,
;; change `org-directory'. It must be set before org loads!
(setq org-directory ``\textasciitilde{}/Documents/.org/'')

;; This determines the style of line numbers in effect. If set to `nil', line
;; numbers are disabled. For relative line numbers, set this to `relative'.
(setq display-line-numbers-type 'relative)

;; Here are some additional functions/macros that could help you configure Doom:
;;
;; - `load!' for loading external *.el files relative to this one
;; - `use-package!' for configuring packages
;; - `after!' for running code after a package has loaded
;; - `add-load-path!' for adding directories to the `load-path', relative to
;;   this file. Emacs searches the `load-path' when you load packages with
;;   `require' or `use-package'.
;; - `map!' for binding new keys
;;
;; To get information about any of these functions/macros, move the cursor over
;; the highlighted symbol at press 'K' (non-evil users must press 'C-c c k').
;; This will open documentation for it, including demos of how they are used.
;;
;; You can also try 'gd' (or 'C-c c d') to jump to their definition and see how
;; they are implemented.

;; ---- PROJECTILE ----

(setq projectile-project-search-path '((``\textasciitilde{}/Documents/Projects'') (``\textasciitilde{}/Documents/Study/'' . 5)))

(defun enable-treemacs-follow-project-minor ()
  ``Enables treemacs-follow-project-mode via function call''
  (setq! treemacs-project-follow-mode 1))

(after! projectile
        (projectile-register-project-type 'latex '(``\textbf{.bib``)
                :project-file ''}.bib''
                :compile ``latexmk -pdflatex=xelatex -shell-escape -bibtex -f -pdf \%f''
                :run ``evince \%f''))

;; ---- UTILS ----

(defun try-init-python-venv ()
  ``Enables python venv if project root contains .venv folder/file''
  (let ((venvdir (concat (projectile-acquire-root) ``.venv'')))
    (if (file-directory-p venvdir)
        (init-python-venv venvdir))))

(defun init-python-venv (venvdir)
  ``Initializes python venv directory''
        (setq pyvenv-mode 1)
        (pyvenv-activate venvdir)
        (print (concat ``Activated python (pyvenv and pipvenv) venv in: '' venvdir)))

(defun magit-push-implicitly-wrapper()
  ``Pushes to the default remote (origin)''
  (magit-push-to-remote nil nil))

;; ---- ORG ----

(after! org
  ;; Pwetty symbols
  (after! org-superstar
    (setq org-superstar-headline-bullets-list '(``◉'' ``○'' ``✸'' ``✿'' ``✤'' ``✜'' ``◆'' ``▶'')
          org-superstar-prettify-item-bullets t ))

(setq org-format-latex-options (plist-put org-format-latex-options :scale 1.1))
;; Collapsed icons
(setq org-ellipsis `` ▾ ''
      org-hide-leading-stars t
      org-priority-highest ?A
      org-priority-lowest ?E
      org-priority-faces
      '((?A . 'all-the-icons-red)
        (?B . 'all-the-icons-orange)
        (?C . 'all-the-icons-yellow)
        (?D . 'all-the-icons-green)
        (?E . 'all-the-icons-blue))))

;; \LaTeX{} syntax highlighing
(setq org-highlight-latex-and-related '(native script entities))
(require 'org-src)
(add-to-list 'org-src-block-faces '(``latex'' (:inherit default :extend t)))

;; Better \LaTeX{} previews
(use-package! org-fragtog
  :hook (org-mode . org-fragtog-mode))

;; helm-bibtex related stuff
(after! helm
  (use-package! helm-bibtex
    :custom
    ;; In the lines below I point helm-bibtex to my default library file.
    ;; (bibtex-completion-bibliography '(``\textasciitilde{}/Documents/.references/refs.bib''))
    ;; (reftex-default-bibliography '(``\textasciitilde{}/Documents/.references/refs.bib''))
    ;; The line below tells helm-bibtex to find the path to the pdf
    ;; in the ``file'' field in the .bib file.
    (bibtex-completion-pdf-field ``file'')
    :hook (Tex . (lambda () (define-key Tex-mode-map ``\C-ch'' 'helm-bibtex))))
  ;; I also like to be able to view my library from anywhere in emacs, for example if I want to read a paper.
  ;; I added the keybind below for that.
  (map! :leader
        :desc ``Open literature database''
        ``o l'' \#'helm-bibtex)
  ;; And I added the keybinds below to make the helm-menu behave a bit like the other menus in emacs behave with evil-mode.
  ;; Basically, the keybinds below make sure I can scroll through my list of references with C-j and C-k.
  (map! :map helm-map
        ``C-j'' \#'helm-next-line
        ``C-k'' \#'helm-previous-line)
  (map! :mode org-mode
        :leader
        :desc ``References''
        ``R i'' \#'org-ref-cite-insert-helm))

;; Set up org-ref stuff
(use-package! org-ref
    :custom
    ;; Again, we can set the default library
    ;; (org-ref-default-bibliography ``\textasciitilde{}/Documents/.references/refs.bib'')
    ;; The default citation type of org-ref is cite:, but I use citep: much more often
    ;; I therefore changed the default type to the latter.
    (org-ref-default-citation-link ``parencite'')
 ;; The function below allows me to consult the pdf of the citation I currently have my cursor on.
 (defun my/org-ref-open-pdf-at-point ()
  ``Open the pdf for bibtex key under point if it exists.''
  (interactive)
  (let* ((results (org-ref-get-bibtex-key-and-file))
         (key (car results))
         (pdf-file (funcall org-ref-get-pdf-filename-function key)))
    (if (file-exists-p pdf-file)
        (find-file pdf-file)
      (message ``No PDF found for \%s'' key)))))

(setq org-ref-completion-library 'org-ref-ivy-cite
      org-export-latex-format-toc-function 'org-export-latex-no-toc
      org-ref-get-pdf-filename-function
      (lambda (key) (car (bibtex-completion-find-pdf key)))
      ;; See the function I defined above.
      org-ref-open-pdf-function 'my/org-ref-open-pdf-at-point
      ;; For pdf export engines.
      ;; org-latex-pdf-process (list ``latexmk -pdflatex='\%latex -shell-escape -interaction nonstopmode' -pdf -bibtex -f -output-directory=\%o \%f'')
      org-latex-pdf-process (list ``latexmk -pdflatex=xelatex -shell-escape -bibtex -f -pdf \%f'')
      ;; I use orb to link org-ref, helm-bibtex and org-noter together (see below for more on org-noter and orb).
      org-ref-notes-function 'orb-edit-notes)

(after! ox-latex
  (add-to-list 'org-latex-classes
               '(``appa''
                 ``$\backslash$\documentclass[stu]{apa7}
                  $\backslash$\usepackage[american]{babel}

$\backslash$\usepackage[backend=biber,style=apa]{biblatex} ``

               ("$\backslash$\section{%s}'' . ``$\backslash$\section*{%s}'')
               (``$\backslash$\subsection{%s}'' . ``$\backslash$\subsection*{%s}'')
               (``$\backslash$\subsubsection{%s}'' . ``$\backslash$\subsubsection*{%s}'')
               (``$\backslash$\paragraph{%s}'' . ``$\backslash$\paragraph*{%s}'')
               (``$\backslash$\subparagraph{%s}'' . ``$\backslash$\subparagraph*{%s}'')))
(add-to-list 'org-latex-classes
             '(``paper''
             ``$\backslash$\documentclass[11pt]{article}
              $\backslash$\usepackage[utf8]{inputenc}
              $\backslash$\usepackage{hyperref}
              $\backslash$\usepackage{setspace}
              $\backslash$\usepackage{palatino}
              $\backslash$\usepackage{graphicx}
              $\backslash$\usepackage{float}
              $\backslash$\usepackage{titling} \% drop vertical space before the title
              $\backslash$\usepackage{multirow}
              $\backslash$\usepackage{lscape}
              $\backslash$\usepackage{amsmath}
              $\backslash$\usepackage{amssymb}
              $\backslash$\usepackage{subcaption}
              $\backslash$\usepackage[a4paper, total=\{6in, 9.5in\}]\{geometry\}
              $\backslash$\fontfamily{ppl}$\backslash$\selectfont

$\backslash$\usepackage[american]{babel}
$\backslash$\usepackage[backend=biber, style=apa]{biblatex}

$\backslash$\usepackage{setspace}
$\backslash$\renewcommand{\\baselinestretch}{1.5}``

               ("$\backslash$\section{%s}'' . ``$\backslash$\section*{%s}'')
               (``$\backslash$\subsection{%s}'' . ``$\backslash$\subsection*{%s}'')
               (``$\backslash$\subsubsection{%s}'' . ``$\backslash$\subsubsection*{%s}'')
               (``$\backslash$\paragraph{%s}'' . ``$\backslash$\paragraph*{%s}'')
               (``$\backslash$\subparagraph{%s}'' . ``$\backslash$\subparagraph*{%s}'')))
(add-to-list 'org-latex-classes
             '(``tudelft\textsubscript{basic}''
               ``$\backslash$\documentclass[english]{article}
               $\backslash$\usepackage{geometry}
               $\backslash$\geometry{verbose,tmargin=3cm,bmargin=3cm,lmargin=3cm,rmargin=3cm}
               $\backslash$\makeatletter
               $\backslash$\usepackage{url}
               $\backslash$\makeatother
               $\backslash$\usepackage[american]{babel}
               $\backslash$\usepackage[backend=biber, style=apa]{biblatex}''

               ("$\backslash$\section{%s}'' . ``$\backslash$\section*{%s}'')
               (``$\backslash$\subsection{%s}'' . ``$\backslash$\subsection*{%s}'')
               (``$\backslash$\subsubsection{%s}'' . ``$\backslash$\subsubsection*{%s}'')
               (``$\backslash$\paragraph{%s}'' . ``$\backslash$\paragraph*{%s}'')
               (``$\backslash$\subparagraph{%s}'' . ``$\backslash$\subparagraph*{%s}'')))
(add-to-list 'org-latex-classes
               '(``tudelft\textsubscript{multicol}''
                 ``$\backslash$\documentclass[english]{article}
                 $\backslash$\usepackage{geometry}
                 $\backslash$\geometry{verbose,tmargin=3cm,bmargin=3cm,lmargin=3cm,rmargin=3cm}
                 $\backslash$\makeatletter
                 $\backslash$\usepackage{url}
                 $\backslash$\makeatother
                 $\backslash$\usepackage[american]{babel}
                 $\backslash$\usepackage[backend=biber, style=apa]{biblatex}
                 $\backslash$\usepackage{multicol}''

(``$\backslash$\section{%s}'' . ``$\backslash$\section*{%s}'')
(``$\backslash$\subsection{%s}'' . ``$\backslash$\subsection*{%s}'')
(``$\backslash$\subsubsection{%s}'' . ``$\backslash$\subsubsection*{%s}'')
(``$\backslash$\paragraph{%s}'' . ``$\backslash$\paragraph*{%s}'')
(``$\backslash$\subparagraph{%s}'' . ``$\backslash$\subparagraph*{%s}''))))

(eval-after-load ``org''
  '(progn
     ;; .txt files aren't in the list initially, but in case that changes
     ;; in a future version of org, use if to avoid errors
     (if (assoc ``$\backslash$\.txt$\backslash$\''' org-file-apps)
         (setcdr (assoc ``$\backslash$\.txt$\backslash$\''' org-file-apps) ``emacs \%s'')
       (add-to-list 'org-file-apps '(``$\backslash$\.txt$\backslash$\''' . ``emacs \%s'') t))
     ;; Change .pdf association directly within the alist
     (setcdr (assoc ``$\backslash$\.pdf$\backslash$\''' org-file-apps) ``evince \%s'')))

;; ---- DART ----

(add-hook 'dart-mode-hook 'lsp)

(setq company-idle-delay 0.2
  company-minimum-prefix-length 2)

(setq lsp-dart-flutter-widget-guides t)
(setq lsp-dart-dap-flutter-hot-restart-on-save t)
(setq lsp-dart-line-length 80)
(setq lsp-dart-closing-labels t)
(setq lsp-dart-closing-labels-prefix ``⇝'')
(setq lsp-dart-flutter-fringe-colors t)
(setq treemacs-indent-guide-mode t)
(setq lsp-dart-sdk-dir ``\emph{home/zohar/snap/flutter/common/flutter/bin/cache/dart-sdk``)
(setq lsp-dart-flutter-sdk-dir ''/home/zohar/snap/flutter/common/flutter}'')

;; ---- JAVA ----

(after! meghanada)
(setq meghanada-java-path ``/usr/lib/jvm/java-11-openjdk/bin/java'')
(setq meghanada-maven-path ``mvn'')

;; ---- DEVENV ----

(defvar-local my/flycheck-local-cache nil)

(defun my/flycheck-checker-get (fn checker property)
  (or (alist-get property (alist-get checker my/flycheck-local-cache))
      (funcall fn checker property)))

(advice-add 'flycheck-checker-get :around 'my/flycheck-checker-get)

(add-hook 'lsp-managed-mode-hook
          (lambda ()
            (when (derived-mode-p 'python-mode)
              (setq my/flycheck-local-cache '((lsp . ((next-checkers . (python-pylint)))))))))

;; ---- PROJECTILE ----

(after! projectile
  (setq projectile-sort-order 'recentf)
  (setq projectile-indexing-method 'alien))

;; ---- PYTHON ----

(add-hook! 'python-mode-hook
  (defun \sout{python-use-correct-flycheck-executables-h ()
    ``Use the correct Python executables for Flycheck.''
    (let ((executable python-shell-interpreter))
      (save-excursion
        (goto-char (point-min))
        (save-match-data
          (when (or (looking-at ``\#!/usr/bin/env $\backslash$\(python[\^{} \n]}$\backslash$\)``)
                    (looking-at ''\#!$\backslash$\([^ \n]+/python[^ \n]+\\)``))
            (setq executable (substring-no-properties (match-string 1))))))
      ;; Try to compile using the appropriate version of Python for
      ;; the file.
      (setq-local flycheck-python-pycompile-executable executable)
      ;; We might be running inside a virtualenv, in which case the
      ;; modules won't be available. But calling the executables
      ;; directly will work.
      (setq-local flycheck-python-pylint-executable ''pylint``)))
  )

(setq-hook! 'python-mode-hook tab-width python-indent-offset)

;; ---- DAP ----

(map! :map dap-mode-map
      :leader
      :prefix (``d'' . ``dap'')
      ;; basics
      :desc ``dap next''          ``n'' \#'dap-next
      :desc ``dap step in''       ``i'' \#'dap-step-in
      :desc ``dap step out''      ``o'' \#'dap-step-out
      :desc ``dap continue''      ``c'' \#'dap-continue
      :desc ``dap hydra''         ``h'' \#'dap-hydra
      :desc ``dap debug restart'' ``r'' \#'dap-debug-restart
      :desc ``dap debug''         ``s'' \#'dap-debug

;; debug
:prefix (``dd'' . ``Debug'')
:desc ``dap debug recent''  ``r'' \#'dap-debug-recent
:desc ``dap debug last''    ``l'' \#'dap-debug-last

;; eval
:prefix (``de'' . ``Eval'')
:desc ``eval''                ``e'' \#'dap-eval
:desc ``eval region''         ``r'' \#'dap-eval-region
:desc ``eval thing at point'' ``s'' \#'dap-eval-thing-at-point
:desc ``add expression''      ``a'' \#'dap-ui-expressions-add
:desc ``remove expression''   ``d'' \#'dap-ui-expressions-remove

:prefix (``db'' . ``Breakpoint'')
:desc ``dap breakpoint toggle''      ``b'' \#'dap-breakpoint-toggle
:desc ``dap breakpoint condition''   ``c'' \#'dap-breakpoint-condition
:desc ``dap breakpoint hit count''   ``h'' \#'dap-breakpoint-hit-condition
:desc ``dap breakpoint log message'' ``l'' \#'dap-breakpoint-log-message)

(after! dap-mode
  (setq dap-python-debugger 'debugpy))

;; ---- HOOKS ----

(add-hook! 'kill-emacs-hook
           \#'magit-push-implicitly-wrapper)

(add-hook! 'python-mode-hook
           \#'try-init-python-venv)

(add-hook! 'markdown-mode-hook
           \#'turn-on-auto-fill)

(add-hook! 'org-mode-hook
           \#'turn-on-auto-fill
           (visual-line-mode 0))

(add-hook! 'emacs-startup-hook
           \#'keychain-refresh-environment
           \#'enable-treemacs-follow-project-minor)

;; ---- UI ----

(setq doom-modeline-height 35)

(after! centaur-tabs
  :ensure t
  :config
   (setq centaur-tabs-style ``bar''
         centaur-tabs-set-bar 'over
         centaur-tabs-height 35
         centaur-tabs-set-icons t
         centaur-tabs-gray-out-icons 'buffer)
   (centaur-tabs-headline-match)
(centaur-tabs-group-by-projectile-project)
   (centaur-tabs-mode t))

;; ---- SPLASH SCREEN ----

(defun doom-dashboard-draw-ascii-emacs-banner-fn ()
  (let* ((banner
          '(``(╯°□°)╯︵ ┻━┻''))
         (longest-line (apply \#'max (mapcar \#'length banner))))
    (put-text-property
     (point)
     (dolist (line banner (point))
       (insert (+doom-dashboard--center
                +doom-dashboard--width
                (concat
                 line (make-string (max 0 (- longest-line (length line)))
                                   32)))
               ``\n''))
     'face 'doom-dashboard-banner)))

(setq +doom-dashboard-ascii-banner-fn \#'doom-dashboard-draw-ascii-emacs-banner-fn)

(remove-hook '+doom-dashboard-functions \#'doom-dashboard-widget-shortmenu)
(add-hook! '+doom-dashboard-mode-hook (hide-mode-line-mode 1) (hl-line-mode -1))
(setq-hook! '+doom-dashboard-mode-hook evil-normal-state-cursor (list nil))
\end{document}
